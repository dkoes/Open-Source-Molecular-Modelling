\section*{Introduction}

Define open source

Reference recent commentary \cite{Karthikeyan_2014}\cite{Gezelter_2015}\cite{Krylov_2015}
Advantages 
Free
Not a black box 
Can improve, extend, and incorporate
Reproducibility (reference royal society report)
Easy distribution (sourceforge, github)
Did we mention free?

Disadvantages
Difficult to commercialize (not a disadvantage from scientific standpoint, but has implications for users)
Support is often limited or nonexistent 
Development depends on active communities of user-developers 
Many packages are primarily developed/supported by a single person (concern about sustainability)
Modeling is a niche area not highly populated by programmers

Methodology for identifying software: click2drug.org, sourceforge, ??

Shoudl also investigate http://cheminformatics.org/
Evaluation criteria: license, development activity rating, usage, number of developers (active)

Categories:
 cheminformatics (putting toolkits in here), QSAR, visualization, docking, protein structure, protein dynamics, virtual screening, ??/
 
 
 How much evaluation are we going to do?  First look for existing articles.
 
 conformer generation - http://pubs.acs.org/doi/abs/10.1021/ci2004658 - but should update with enhancements to rdkit
 pharmacophore search - http://pubs.acs.org/doi/abs/10.1021/ci2005274
 docking - vina compared itself to autodock..
  
  
  \cite{Koes_2011}
  
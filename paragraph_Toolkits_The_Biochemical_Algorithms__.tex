\paragraph{Toolkits}
The Biochemical Algorithms Library (BALL) \cite{Hildebrandt_2010} provides an object-oriented C++ library for structural bioinformatics, and its capabilities include molecular mechanics, support for reading and writing a variety of file formats, protein-ligand scoring, docking, and QSAR modeling.

The Chemistry Development Kit (CDK) \cite{Steinbeck_2006} is a cheminformatics toolkit written in Java.  Its capabilities include support for reading and writing a variety of chemical formats, descriptor and fingerprint calculation, force field calculations, substructure search, and structure generation.

Chem$^f$ \cite{H_ck_2012} is a minimal cheminformatics toolkit written in the functional language Scala.

ChemmineR \cite{Cao_2008}  is a cheminformatics package for the R statistical programming languages that is built using OpenBabel. Its capabilities include property calculations, similarity search, and classification and clusters of compounds.

Cinfony \cite{cinfony} provides a single, simple standardized interface to other cheminformatics toolkits, including Open Babel, RDKit, the CDK, Indigo, JChem, OPSIN, and several web services.

CurlySMILES \cite{Drefahl_2011} provides parsing functionality for an extension of the SMILES format that supports the description of complex molecular systems.

DisCuS (Database System for Compound Selection) \cite{W_jcikowski_2014} provides support for analyzing the results of a high throughput screen.

Fafoom (flexible aalgorithm for optimization of molecules) \cite{Supady_2015} is a Python library for identifying low energy conformers using a genetic algorithm.

fmcsR \cite{Goecks_2010} is an R package that efficiently performs flexible maximum common substructure matching that allows minor mismatches between atoms and bonds in the common substructure.

Frowns is a cheminformatics toolkit mostly written in Python that provides basic support for SMILES and SD files, SMARTS search, fingerprint generation, and property perception.

Helium is a cheminformatics toolkit written using modern C++ idioms that provides support for SMILES files, fingerprints generation, and SMARTS and SMIRKS.

Indigo \cite{Pavlov_2011} is a cheminformatics toolkit written in C++ with C, Python, Java (including a KNIME node), and C\# bindings.  Its capabilities include general support for manipulating molecules, property calculation, combinatorial chemistry, scaffold detection and R-group decomposition, reaction processing, and substructure matching and similarity search.

JOELib is a cheminformatics toolkit written in Java. Its capabilities include SMARTS substructure search, descriptor calculation, and processing/filtering pipes.

LICSS \cite{Lawson_2012} integrates with the CDK to provide representations and analysis of chemical data embedded within Microsoft Excel.

MayChemTools is a collection of Perl scripts for manipulating chemical data, interfacing with databases, generating fingerprints, performing similarity search, and computing molecular properties.

Mychem is built using OpenBabel and provides an extension to the MySQL database package that adds the ability to search, analyze, and convert chemical data within a MySQL database.

The Open Drug Discovery Toolkit (ODDT) \cite{W_jcikowski_2015} is entirely written in Python, is built on top of RDKit and OpenBabel, and is focused on providing enhanced functionality for managing and implementing drug discovery workflows, such as making it easy to implement a docking pipeline. 

Open Babel \cite{O_Boyle_2011} is substantial cheminformatics toolkit written in C++ with Python, Perl, Java, Ruby, R, PHP, and Scala bindings.  Its capabilities include support for more than 100 chemical file formats, fingerprint generation, property determination, similarity and substructure search, structure generation, and molecular force fields.  It has absorbed the Confab \cite{confab} conformer generator which produces 3D structures through the systematic enumeration of torsions and energy minimization.

OPSIN \cite{Lowe_2011}, the Open Parser for Systematic IUPAC nomenclature, converts plain-text chemical nomenclature to machine readable CML or InChi formats.

OrChem is built using the CDK and provides an extension to Oracle databases that adds the ability to incorporate and search chemical data within an Oracle database.

OSRA \cite{Filippov_2009} provides optical structure recognition. It takes as input an image and generates a SMILES string.

Ouch (Ouch Uses Chemical Haskell) is a minimal cheminformatics toolkit written in the functional language Haskell.

Pybel \cite{pybel} provides the full functionality of Open Babel, but common routines are provided in a simplified, more `pythonic' interface.

RDKit is a substantial cheminformatics toolkit written in C++ with Python, Java and C\# bindings.  Its capabilities include file handling, manipulation of molecular data, chemical reactions, substantial support for fingerprinting, substructure and similarity search, 3D conformer generation, property determination, force field support, shape-based alignment and screening, and integration with PyMOL, KNIME, and PostgreSQL.

rubabel \cite{Smith_2013} is similar to Pybel in that it provides a native Ruby interface to Open Babel.

The Small Molecule Subgraph Detector (SMSD) \cite{Rahman_2009} is a Java library for calculating the maximum common subgraph between small molecules.

Som-it\textsuperscript{TM}  is an R package for creating and visualizing self-organizing maps from large datasets.


\paragraph{Standalone Programs}

The utilities checkmol and matchmol \cite{Haider_2010} decompose and compare functional groups of input molecules.

ConvertMAS is a utility for converting between formates and merging and splitting multi-molecule files.

Filter-it\textsuperscript{TM} filters a set of molecules based on their properties such as physicochemical parameters and graph-based properties. 

Frog2 \cite{Miteva_2010} uses a two stage Monte Carlo approach coupled with energy minimization to rapidly generate conformers.

The Lilly MedChem Rules (LMR)  \cite{Bruns_2012} apply filters to avoid reactive and promiscuous compounds.

Molpher  \cite{Hoksza_2014} generates a virtual chemical library that represents the chemical space between two input molecules as it consists of the path found by morphing one molecule to another.

MoSS (Molecular Subsstructure miner) \cite{Borgelt_2005} finds common molecular substructures and discriminative fragments within a compound library.

The Open Molecule Generator (OMG) \cite{Peironcely_2012} enumerates all possible chemical structures given constraints on their composition.

sdf2xyz2sdf  \cite{Tosco_2011} converts between SDF and TINKER XYZ files.

sdsorter provides convenient routines for manipulating, sorting, and filtering the contents of sdf molecular data files based on the embedded sd data tags.

Shape \cite{Rosen_2009} employs a genetic algorithm to generate conformations of carbohydrates.

Strip-it\textsuperscript{TM} is built using Open Babel and extracts molecular scaffolds. 


\paragraph{Graphical Development Environments}

Ambit \cite{Jeliazkova_2011} integrates with CDK to provide web-based applications for chemical search and analysis.

Bioclipse  \cite{Spjuth_2009} is a workbench, based on the Eclipse framework, for manipulating and analyzing biochemical data and databases. It integrates with CDK and Jmol to provide cheminformatic functionality and also has modules for bioinformatics (primarly sequence analysis) and QSAR modeling.

Galaxy \cite{Goecks_2010} is a web platform for exploring biomedical data and includes as a component a Chemical Toolbox that integrates a number of other cheminformatics tools to offer an array of molecular search, property calculation, clustering, and manipulations capabilities.

The Konstanz Information Miner (KNIME) is a general workflow environment that includes a number of plugins for cheminformatics, such as CDK \cite{Beisken_2013} and RDKit modules, as well as bioinformatics and machine learning modules.

Screening Assistant 2 (SA2) \cite{Guilloux_2012} is a GUI written in Java that integrates with other toolkits to help manage, analyze, and visualize libraries of compounds.

Weka \cite{Hall_2009} is a platform for data mining and machine learning that can be adapted for cheminformatics.  

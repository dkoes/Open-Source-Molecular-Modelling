\paragraph{Toolkits}
The Biochemical Algorithms Library (BALL) \cite{Hildebrandt_2010} provides an object-oriented C++ library for structural bioinformatics, and its capabilities include molecular mechanics, support for reading and writing a variety of file formats, protein-ligand scoring, docking, and QSAR modeling.

The Chemistry Development Kit (CDK) \cite{Steinbeck_2006} is a cheminformatics toolkit written in Java.  Its capabilities include support for reading and writing a variety of chemical formats, descriptor and fingerprint calculation, force field calculations, substructure search, and structure generation.

Chem$^f$ \cite{H_ck_2012} is a minimal cheminformatics toolkit written in the functional language Scala.

ChemmineR \cite{Cao_2008}  is a cheminformatics package for the R statistical programming languages that is built using OpenBabel. Its capabilities include property calculations, similarity search, and classification and clusters of compounds.

ConvertMAS is a utility for converting between formates and merging and splitting multi-molecule files.

CurlySMILES \cite{Drefahl_2011} provides parsing functionality for an extension of the SMILES format that supports the description of complex molecular systems.

Filter-it\textsuperscript{TM} filters a set of molecules based on their properties such as physicochemical parameters and graph-based properties. 

fmcsR \cite{Goecks_2010} is an R package that efficiently performs flexible maximum common substructure matching that allows minor mismatches between atoms and bonds in the common substructure.

Frowns is a cheminformatics tookkit mostly written in Python that provides basic support for SMILES and SD files, SMARTS search, fingerprint generation, and property perception.

Helium is a cheminformatics toolkit written using modern C++ idioms that provides support for SMILES files, fingerprints generation, and SMARTS and SMIRKS.

Indigo \cite{Pavlov_2011} is a cheminformatics toolkit written in C++ with C, Python, Java (including a KNIME node), and C# bindings.  Its capabilities include general support for manipulating molecules, property calculation, scaffold detection and R-group decomposition, reaction processing, and substructure matching and similarity search.

JOELib is a cheminformatics toolkit written in Java. Its capabilities include SMARTS substructure search, descriptor calculation, and processing/filtering pipes.

LICSS \cite{Lawson_2012} integrates with the CDK to provide representations and analysis of chemical data embedded within Microsoft Excel.

MayChemTools is a collection of Perl script for manipulating chemical data, interfacing with databases, generating fingerprints, performing similarity search, and computing molecular properties.


\paragraph{Graphical Development Environments}

Ambit \cite{Jeliazkova_2011} integrates with CDK to provide web-based applications for chemical search and analysis.

Bioclipse  \cite{Spjuth_2009} is a workbench, based on the Eclipse framework, for manipulating and analyzing biochemical data and databases. It integrates with CDK and Jmol to provide cheminformatic functionality and also has modules for bioinformatics (primarly sequence analysis) and QSAR modeling.

Galaxy \cite{Goecks_2010} is a web platform for exploring biomedical data and includes as a component a Chemical Toolbox that integrates a number of other cheminformatics tools to offer an array of molecular search, property calculation, clustering, and manipulations capabilities.

The Konstanz Information Miner (KNIME) is a general workflow environment that includes a number of plugins for cheminformatics, such as CDK \cite{Beisken_2013} and RDKit modules, as well as bioinformatics and machine learning modules.


\paragraph{Molecular Editors}

BKChem is a 2D molecular editor written in python that uses the Tk GUI toolkit.

Chemtool is a 2D molecular editor for Linux systems that uses the GTK toolkit.

Molsketch is a 2D molecular editor written in C++ with the Qt toolkit that includes support for the Windows and Android operating systems.

\paragraph{Conformer Generation}


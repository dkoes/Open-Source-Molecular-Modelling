\paragraph{Docking}

ADplugin is a plugin for PyMOL for interfacing with AutoDock and AutoDock Vina.

AutoDock \cite{Morris_2009} is an automated docking program that uses a physics-based semiempirical scoring function \cite{Huey_2007} mapped to atom type grids to evaluate poses and a genetic algorithm to explore the conformational space.  It includes the ability to incorporate sidechain flexibility and covalent docking.

AutoDock Vina \cite{Trott_2009} is an entirely separate code base and approach from Autodock that was developed with a focus runtime performance and ease of system setup. It uses a fully empirical scoring function and an iterated local search global optimizer to produce docked poses. It includes support for multi-threading and flexible sidechains.

Clusterizer-DockAccessor \cite{Ballante_2016} are tools for accessing the quality of docking protocols. It interfaces with a number of open-source and free tools.

DockoMatic \cite{Bullock_2013}  provides a graphical user interface for setting up and analyzing AutoDock and AutoDock Vina docking jobs, including when run on a cluster. It also includes the ability to run inverse virtual screens (find proteins that bind a given ligand) and support for homology model construction.

DOVIS \cite{Jiang_2008} is an extension of AutoDock 4.0 that provides more efficient parallelization of large virtual screening jobs.

idock \cite{Li_2012} is a multi-threaded docking program that includes support for the AutoDock Vina scoring function and a random forest scoring function. I can output per-atom free energy information for hotspot detection.

MOLA \cite{Abreu_2010} is a pre-packaged distribution of AutoDock and AutoDock Vina for deployment on multi-platform computing clusters.

NNScore \cite{Durrant_2011} uses a neural network model to score protein-ligand poses.

Paradocks \cite{Meier_2010} is a parallelized docking program that includes a number of population-based metaheuristics, such as particle swarm optimization, for exploring the 

PyRx \cite{Dallakyan_2014} is a visual interface for AutoDock and AutoDock Vina that simplifies setting up and analyzing docking workflows.  Its future as an open-source solution is in doubt due to a recent shift to commercialization.

rDock \cite{Ruiz_Carmona_2014}  is designed for docking against proteins or nucleic acids and can incorporate user-specified constraints. It uses an empirical scoring function that includes solvent accessible surface area terms. A combination of genetic algorithms, Monte Carlo, and simplex minimization is used to explore the conformational space. Distinct scoring functions are provided for docking to proteins and nucleic acids.

RF-Score \cite{Li_2015,Ballester_2010} uses a random forest classifier to score protein-ligand poses.

smina \cite{Koes_2013} is a fork of AutoDock Vina designed to better support energy minimization and custom scoring function development (scoring function terms and atom type properties can be specified using a run-time configuration file). It also simplifies the process of setting up a docking run with flexible sidechains.

VinaLC  \cite{Zhang_2013} is a fork of AutoDock Vina designed to run on a cluster of multiprocessor machines.

VinaMPI \cite{Ellingson_2013} is a wrapper for AutoDock Vina that uses OpenMPI to run large-scale virtual screens on a computing cluster.

Zodiac \cite{Zonta_2008} is a visual interface for structure-based drug design that includes support for haptic feedback.

\paragraph{Pocket Detection}
eFindSite \cite{Brylinski_2013} using homology modeling and machine learning to predict ligand binding sites in a protein structure.

fpocket  \cite{Schmidtke_2011} detects and delineates protein cavities using Voronoi tessellation and is able to process molecular dynamics simulations.

KVFinder \cite{Oliveira_2014} is a PyMOL plugin for identifying and characterizing pockets.

PAPCA (PocketAnalyzerPCA) is a pocket detection utility designed to analyze ensembles of protein conformations.

PCS (Pocket Cavity Search) measures the volume of internal cavities and evaluates the environment of ionizable residues.

PocketPicker \cite{Weisel_2007}  is a PyMOL plugin that automatically identifies potential ligand binding sites using a grid-based shape descriptor.

POVME (POcket Volume MEasurer) \cite{Durrant_2014} is tool for measuring and characterizing pocket volumes that includes a graphical user interface.

mcvol \cite{Till_2009} calculates protein volumes and identifying cavities using a Monte Carlo algorithm.

VHELIBS (Validation HElper for LIgands and Binding Sites) \cite{Cereto_Massagu__2013} assists the non-crystallographer in validating ligand geometries with respect to electron density maps.


\paragraph{Screening}

Pharmer \cite{Koes_2011} uses efficient data structures to rapidly screen large libraries for ligand conformations that match a pharmacophore.

Pharmit \cite{Sunseri_2016} is an extension of Pharmer that also incorporates shape matching and energy minimization as part of the screen.  It is primarily intended to be used as a backend to a web service.
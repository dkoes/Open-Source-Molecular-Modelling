AutoDock \cite{Morris_2009} is an automated docking program that uses a physics-based semiempirical scoring function \cite{Huey_2007} mapped to atom type grids to evaluate poses and a genetic algorithm to explore the conformational space.  It includes the ability to incorporate sidechain flexibility and covalent docking.

AutoDock Vina \cite{Trott_2009} is an entirely separate code base and approach from Autodock that was developed with a focus runtime performance and ease of system setup. It uses a fully empirical scoring function and an iterated local search global optimizer to produce docked poses. It includes support for multi-threading and flexible sidechains.

MOLA \cite{Abreu_2010} is a pre-packaged distribution of AutoDock and AutoDock Vina for deployment on multi-platform computing clusters.

PyRx \cite{Dallakyan_2014} is a visual interface for AutoDock and AutoDock Vina that simplifies setting up and analyzing docking workflows.  Its future as an open-source solution is in doubt due to a recent shift to commercialization.

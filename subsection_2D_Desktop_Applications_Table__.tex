\subsection*{2D Desktop Applications (Table~\ref{2ddesktopviz})}

BKChem is a 2D molecular editor written in python that uses the Tk GUI toolkit.

chemfig is a tool for embedding chemical drawings in {\LaTeX} documents.

Chemtool is a 2D molecular editor for Linux systems that uses the GTK toolkit.

JChemPaint \cite{Krause_2000} is a Java-based 2D molecular editor built using the CDK toolkit.

LeView \cite{Caboche_2013} generates 2D representations of ligand-protein interactions that highlight features such as hydrogen bonds.

mol2chemfig \cite{Brefo_Mensah_2012} converts SMILES files into {\LaTeX} source code.

Molsketch is a 2D molecular editor written in C++ with the Qt toolkit that includes support for the Windows and Android operating systems.

SketchEl is a Java-based 2D molecular editor that includes support for a datasheet view for handling multi-molecule files.

\subsection*{3D Desktop Applications (Table~\ref{3ddesktopviz})}

Avogadro \cite{Hanwell_2012} is a 3D molecular viewer and editor with a modular plugin architecture with both Python and C++ bindings that includes interactive structure optimization for real-time editing. 

BALLView \cite{Moll_2005} provides interactive 3D visualizations as part of the BALL \cite{Hildebrandt_2010} cheminformatics toolkit.

gMol provides basic interactive 3D visualizations of molecular data readable by Open Babel.

Jamberoo provides a basic Java-based 3D molecular viewer and editor.

LP Molecular Viewer is an ActiveX/ATL control for embedding interactive 3D representations of molecular data in Microsoft products.

Luscus \cite{Kova_evi__2015} is a 3D viewer and editor that is designed with a focus on electronic structure information.

OpenStructure \cite{Biasini_2013} is a computational structural biology framework that provides a 3D viewer for manipulating structural information and includes an interactive Python shell.

PLIP (Protein-Ligand Interaction Profiler) \cite{Salentin_2015} runs as a web application and analyzes and visualizes protein-ligand interactions in 3D.

PyMOL is a substantial 3D molecular viewer that includes a full Python interface to support scripting and plugin development.

RasTop and OpenRasMol are based off the venerable RasMol software and provide basic 3D visualization. 

SPADE (Structural Proteomics Application Development Environment) \cite{sweeney2011computational} is a graphical Python interface for structural informatics.

QuteMol \cite{Tarini_2006} provides high-quality, visually engaging renderings of 3D molecular data.


\subsection*{Web-Based Visualization (Table~\ref{webviz})}

3Dmol.js \cite{Rego_2014} is a JavaScript library that provides WebGL-accelerated interactive 3D visualizations of molecular structures and surfaces.

CH5M3D \cite{Earley_2013} uses JavaScript and HTML5 to provide visualization and editing of 3D structures of small molecules.

Chemozart \cite{Mohebifar_2015} is a WebGL-based web application for 3D editing of small molecules.

CWC (ChemDoodle Web Components) \cite{Burger_2015} provides a suite of web-based visualizers and editors for 2D and 3D molecular data.

JSME \cite{Bienfait_2013} is a pure JavaScript 2D molecular editor that can export and import SMILES data.

Jmol \cite{Hanson_2010} is a Java applet for interactive 3D visualization that provides significant cheminformatics support and a custom scripting language.

JSmol \cite{Hanson_2013} is the JavaScript port of Jmol and does not require the Java plugin to run.

NGL \cite{Rose_2015} is a WebGL-accelerated viewer and JavaScript library for interactive 3D visualization of macromolecules.

PV (Protein Viewer) \cite{95f13b46-4e83-4cdd-afc0-6de07bca5ae8} is a WebGL-accelerated viewer for interactive 3D visualization of macromolecules with a functional-style API.





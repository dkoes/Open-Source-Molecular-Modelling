\subsection*{Simulation Software (Table~\ref{mdtable})}

Campari \cite{Vitalis_2009} conducts flexible Monte Carlo sampling of biopolymers in internal coordinate space, with built-in analysis routines to estimate structural properties and support for replica exchange and Wang-Landau sampling.

DL\_POLY Classic \cite{Smith_2002} is a general purpose molecular dynamics simulation package that can run in parallel and includes a Java graphical user interface.

GALAMOST (GPU accelerated large-scale molecular simulation toolkit) \cite{Zhu_2013} uses GPU computing to perform traditional molecular dynamics with a special focus on polymeric systems at mesoscopic scales.

Gromacs \cite{Pronk_2013} is a complete and well-established package for molecular dynamics simulations that provides high performance on both CPUs and GPUs.  It can be used for free energy and QM/MM calculations and includes a comprehensive set of analysis tools.

Iphigenie \cite{Lorenzen_2012} is a molecular mechanics program that features polarizable force fields, the HADES reaction field, and QM/(P)MM hybrid simulations.

LAMMPS  (Large-scale Atomic/Molecular Massively Parallel Simulator) \cite{Plimpton_1995} is a highly modular classical molecular dynamics simulator that includes a diverse array of energy potentials and integrators.

MDynaMix \cite{Lyubartsev_2000} is a basic general purpose molecular dynamics package.

MMTK (Molecular Modelling Toolkit) \cite{Hinsen_2000} is a library written in Python (with some time critical parts written in C) for constructing and simulating molecular systems.  Its capabilities include molecular dynamics, energy minimization, and normal mode analysis and it is well-suited for methods development.

OpenMM  \cite{Eastman_2013} is a substantial toolkit for high performance molecular dynamics simulations that includes support for GPU acceleration.

ProtoMol \cite{Matthey_2004}, and the associated MDLab Python bindings \cite{Cickovski_2009}, provides an object-oriented framework for prototyping algorithms for molecular dynamics simulations and includes an interface to OpenMM.

ProtoMS \cite{Michel_2006} is a Monte Carlo biomolecular simulation program which can be used to calculate relative and absolute free energies and water placement with the GCMC and JAWS methodologies.

Sire is a collection of modular libraries intended to facilitate fast prototyping and the development of new algorithms for molecular simulation and molecular design. It has apps for system setup, simulation, and analysis.

WESTPA  (The Weighted Ensemble Simulation Toolkit with Parallelization and Analysis) \cite{Zwier_2015} is a library for performing weighted ensemble simulations to sample rare events and compute rigorous kinetics.

yank is built off of OpenMM and provides a Python interface for performing alchemical free energy calculations.

\subsection*{Simulation Setup and Analysis (Table~\ref{mdanalysis})}
AmberTools \cite{Salomon_Ferrer_2012} is an open source component of the non-open source Amber package and provides a large suite of analysis programs. As of Amber15, AmberTools includes the lower performance, but readily extendable,  sander molecular dynamics code.

LOOS (Lightweight Object-Oriented Structure library) \cite{Romo_2014} is a C++ library (with Python bindings) for reading and analyzing molecular dynamics trajectories that also includes a number of standalone programs.

lsfitpar \cite{Vanommeslaeghe_2015} derives bonded parameters for Class I force fields by performing a robust fit to potential energy scans provided by the user.

MDAnalysis  \cite{Michaud_Agrawal_2011} is a Python library for reading and analyzing molecular dynamics simulations with some time critical sections written in C.

MDTraj \cite{McGibbon_2015} provides high-performance reading, writing, and analysis of molecular dynamics trajectories in a diversity of formats from a Python interface.

MEMBPLUGIN \cite{Guixa-Gonzalez_2014} analyzes molecular dynamics simulations of lipid bilayers and is most commonly used as a VMD plugin.

MEPSA (Minimum Energy Pathway Analysis) \cite{Marcos_Alcalde_2015} provides tools for analyzing energy landscapes and pathways.

MSMBuilder \cite{Beauchamp_2011} is an application and Python library for building Markov models of high-dimensional trajectory data.

packmol \cite{Mart_nez_2009} is a utility for setting molecular systems  by realistically packing molecules to obey a variety of constraints and can create solvent mixtures and lipid bilayers.

PDB2PQR \cite{Dolinsky_2007} prepares structures for electrostatics calculations by adding hydrogens, calculating sidechain pKa, adding missing heavy atoms, and assigning force field-dependent parameters; users can specify an ambient pH.

PLUMED \cite{Tribello_2014} interfaces with an assortment of molecular dynamics software packages to provide a unified interface for performing free energy calculations using methods such as metadynamics, umbrella sampling and steered MD (Jarzynski).

ProDy \cite{Bakan_2011} is a Python toolkit for analyzing proteins and includes facilities for trajectory analysis and druggability predictions using simulations of molecular probes \cite{Bakan_2012}.

Pteros \cite{Yesylevskyy_2015} is a C++ library (with Python bindings) for reading and analyzing molecular dynamics trajectories.

PyEMMA \cite{Scherer_2015} is a Python library for performing kinetic and thermodynamic analyses of molecular dynamics simulations using Markov models. 

PyRED \cite{Dupradeau_2010} generates RESP and ESP charges for the AMBER, CHARMM, OPLS, and Glycam and force fields.

WHAM (Weighted Histogram Analysis Method) calculates the potential of mean force (PMF) from umbrella sampling simulations.

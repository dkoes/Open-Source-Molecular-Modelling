\section{Methods}

We organize software packages into eight categories: cheminformatics, QSAR (quantitative structure–activity relationship software), quantum chemistry, protein structure modeling, protein dynamics modeling, virtual screening, de novo ligand design, and visualization.
We identified open-source software package by browsing the relevant categories (Molecular Science, Chemistry, Bio-Informatics, Medical Science) on the popular SourceForge (\href{http://sourceforge.net}) repository, searching for categories on GitHub (\href{http://github.com}), searching for categories together with ``open-source software'' on Google, and browsing the Click2Drug (\url{http://click2drug.org}) and VLS3D \cite{Villoutreix_2013} directories.

For every identified software package, we report its primary URL and software license and assign it an activity code. Activity codes consist of a development activity level (alphabetical) and usage activity level (numerical). In cases where download statistics are not available/applicable usage is subjectively ranked based on external references to the tool (e.g., message list posts, source code forks, citations, etc.).
\begin{description}
  \item[A] Substantial development (e.g. addition of new features or substantial refinements of existing features) within the last 18 months.
  \item[B] Evidence of some development within the last 18 months.
  \item[C] No evidence of development (changes to the source code or documentation) within the last 18 months.
\end{description}
\begin{enumerate}
  \item Substantial user usage within the last 18 months (more than 10 downloads in the last month and/or clearly active user community).
  \item Moderate user usage within the last 18 months.
    \item Minimal user usage within the last 18 months (fewer than 50 downloads total and/or one or two requests for support or external references).
  \item No evidence of user interest (e.g. posts to a message board or discussion list that are answered, not necessarily by the developer) within the last 18 months.
\end{enumerate}

We omit some packages with extended periods of inactivity where there is little evidence of any usage (e.g. downloads).  We also omit packages that provide common and/or trivial functionality (e.g. molecular weight calculators).  



 
 

  
  
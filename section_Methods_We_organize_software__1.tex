\section{Methods}

We organize software packages into eight categories: cheminformatics, QSAR (quantitative structure–activity relationship software), quantum chemistry, protein structure modeling, protein dynamics modeling, virtual screening, de novo ligand design, and visualization.
We identified open-source software package by browsing the relevant categories (Molecular Science, Chemistry, Bio-Informatics, Medical Science) on the popular SourceForge (\href{http://sourceforge.net}) repository, searching for categories on GitHub (\href{http://github.com}), and searching for categories together with ``open-source software'' on Google.

For every identified software package, we report its primary URL and software license and assign it an activity score. Activity scores are assigned as follows: 
\begin{enumerate}
  \setcounter{enumi}{0}
  \item No evidence of user support (e.g. posts to a message board or discussion list that are answered, not necessarily by the developer) or development (changes to the source code) within the last 18 months.
  \item Evidence of user support but not development within the last 18 months.
  \item Evidence of development within the last 18 months.
  \item Substantial development (e.g. addition of new features or substantial refinements of existing features) within the last 18 months.
  \item Substantial development and usage (e.g. downloads, user feedback).
\end{enumerate}

We omit some packages with extended periods of inactivity where there is little evidence of any usage (e.g. downloads).  We also omit packages that provide common and/or trivial functionality (e.g. molecular weight calculators).



 
 

  
  
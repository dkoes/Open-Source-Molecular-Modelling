\section{Virtual Screening and Ligand Design}

The goal of virtual, or \textit{in silico}, screening is to computationally identify small molecules in a compound library that are active against a given target.  Virtual screening methods usually adopt either a ligand-based approach, where properties of known active compounds are used to identify additional compounds, or a structure-based approach, where the interactions between putative ligands and the receptor structure are used.  Tools for chemical similarity, which can be used for ligand-based screening, are cataloged in the Cheminformatics section.

In contrast to virtual screening, which evaluates predetermined compounds, \textit{de novo} ligand design attempt to create a molecule `from scratch' that binds to a protein.  Methods differ in how the specific the objective to optimize (e.g., docking score to a protein) and how candidate molecules are created, where a key challenge is maintaining synthetic accessibility. 
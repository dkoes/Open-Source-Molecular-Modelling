\section{Methods}

We organize software packages into eight categories: cheminformatics, QSAR (quantitative structure–activity relationship software), quantum chemistry, protein structure modeling, protein dynamics modeling, virtual screening, de novo ligand design, and visualization.
We identified open-source software package by browsing the relevant categories (Molecular Science, Chemistry, Bio-Informatics, Medical Science) on the popular SourceForge (\href{http://sourceforge.net}) repository, searching for categories on GitHub (\href{http://github.com}), and searching for categories together with ``open-source software'' on Google.

For every identified software package, we determine its license and assign it a qualitative activity score. Activity scores are assigned as follows: 
\begin{enumerate}
  \setcounter{enumi}{-1}
  \item No evidence of user support (e.g. posts to a message board or discussion list that are answered, not necessarily by the developer) or development (changes to the source code) within the last 18 months.
  \item Evidence of user support but not development within the last 18 months.
  \item Evidence of development within the last 18 months.
  \item Substantial development (e.g. addition of new features or substantial refinements of existing features) within the last 18 months.
\end{enumerate}

Packages where we find no evidence of user interaction or development within the last year receive a score of 0, if there is evidence of user interaction (e.g., posts to a message board or discussion list that are answered, not necessarily by the developer) then a score of 

Shoudl also investigate http://cheminformatics.org/
Evaluation criteria: license, development activity rating, usage, number of developers (active)


 
 
 How much evaluation are we going to do?  First look for existing articles.
 
 conformer generation - http://pubs.acs.org/doi/abs/10.1021/ci2004658 - but should update with enhancements to rdkit
 pharmacophore search - http://pubs.acs.org/doi/abs/10.1021/ci2005274
 docking - vina compared itself to autodock..
  
  
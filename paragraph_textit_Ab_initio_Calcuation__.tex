\paragraph{\textit{Ab initio} Calcuation (Table~\ref{abinitio})}

ABINIT \cite{Gonze_2009} is a quantum package which is able to calculate the total energy, charge density and electronic structure of molecules and periodic solids within density functional theory (DFT) and Many-Body Perturbation Theory (MBPT), using pseudopotentials and a planewave or wavelet basis. ABINIT also can optimize the geometry, perform molecular dynamics simulations, or generate dynamical matrices, Born effective charges, and dielectric tensors and many more properties. 

ACES \cite{Lotrich_2008} is a collection of programs that perform \textit{ab initio} quantum chemistry calculations such as single point energy calculations, analytical gradients, and analytical Hessians, and is highly parallelized, including support for GPU computing.
A focus of ACES is the use of MBPT and the coupled-cluster approximation to reliable treat electron correlation.

BigDFT \cite{Genovese_2008,Mohr_2014,Mohr_2015} 1performs \textit{ab initio} calculations using Daubechies wavelets and has the capability to use a linear scaling method.  Periodic systems, surfaces and isolated systems can be simulated with the proper boundary conditions. It is included as part of ABINIT.

CP2K \cite{Hutter_2013} performs simulations of solid state, liquid, molecular and biological systems. Its particular focus is massively parallel and linear scaling electronic structure methods and state-of-the-art \textit{ab-initio} molecular dynamics (AIMD) simulations. It is optimized for the mixed Gaussian and Plane-Waves method using pseudopotentials and can run on parallel and on GPUs.

Dacapo is a total energy program that uses density functional theory. It can do molecular dynamics/structural relaxation while solving the Schrodinger equations. It has support for parallel execution and is used through the Atomic Simulation Environment (ASE) \cite{Bahn_2002}

ErgoSCM \cite{Rudberg_2011} is a quantum chemistry program for large-scale self-consistent field calculations. It performs electronic structure calculations using Hartree-Fock and Kohn-Sham density functional theory and achieves linear scaling for both CPU usage and memory utilization.

ERKALE \cite{Lehtola_2012} is designed to compute X-ray properties, such as ground-state electron momentum densities and Compton profiles, and core (x-ray absorption and x-ray Raman scattering) and valence electron excitation spectra of atoms and molecules.

GPAW \cite{gpaw} is a DFT code that uses the projector-augmented wave (PAW) technique \cite{Bl_chl_1994,Kresse_1999} and integrates with the  atomic simulation environment (ASE)  \cite{Bahn_2002}. 

Helpful Open-source Research TOol for N-fermion systems (HORTON) has as a primary design goal ease of extensibility for researching new methods in \textit{ab initio} electronic structure theory.

JANPA \cite{Nikolaienko_2014} computes natural atomic orbitals from a reduced one-particle density matrix.

MPQC (massively parallel quantum chemistry program) \cite{Janssen95} offers many features including closed shell, unrestricted and general restricted open shell Hartree-Fock energies and gradients, closed shell, unrestricted and general restricted open shell density functional theory energies and gradients, second order open shell perturbation theory and Z-averaged perturbation theory energies.

NWChem \cite{Valiev_2010} provides a full suite of methods for modeling both classical and QM systems. Its capabilities include molecular electronic structure, QM/MM, pseudopotential plane-wave electronic structure, and molecular dynamics and is designed to scale across hundreds of processors.

Octopus pervorms \textit{ab initio} calculations using time-dependent DFT (TDDFT) and pseudopotentials.  Included in the project is libxc \cite{Marques_2012} which is a standalone library of exchange-correlation functionals for DFT (released under the LGPL).

OpenMX (Open source package for material eXplorer) \cite{Ozaki_2005} is designed for nano-scale material simulations based on DFT, norm-conserving pseudopotentials, and pseudo-atomic localized basis functions. OpenMX is capable of performing calculations of physical properties such as magnetic, dielectric, and electric transport properties and is optimized for large-scale parallelism.

Psi4 \cite{Turney_2011} is a suite of \textit{ab initio} quantum chemistry programs that supports a wide range of computations (e.g., Hartree–Fock, MP2, coupled-cluster) and general procedures such as geometry optimization and vibrational frequency analysis with more than 2500 basis functions running serially or in parallel.

PyQuante is a collection of modules, mostly written in Python, for performing Hartree-Fock and DFT calculations with a focus on providing a well-engineered set of tools. A new version is under development (\url{https://github.com/rpmuller/pyquante2}).

PySCF is also written primarily in Python and supports several popular methods such as Hartree-Fock, DFT, and MP2. It also has easy of use and extension as primary design goals.

QMCPACK \cite{kim2010quantum} is a many-body \textit{ab initio} quantum Monte Carlo implementation for computing electronic structure properties of molecular, quasi-2D and solid-state systems. The standard file formats utilized for input and output are in XML and HDF5.

QUANTUM ESPRESSO \cite{Giannozzi_2009} is designed for modeling at the nanoscale using DFT, plane waves, and pseudopotentials and its capabilities include ground-state calculations, structural optimization, transition states and minimum energy paths, \textit{ab initio} molecular dynamics, DFT perturbation theory, spectroscopic properties, and quantum transport.

RMG \cite{moore2012scaling} is a DFT code that uses real space grids to provide high scalability across thousands of processors and GPU acceleration for both structural relaxation and molecular dynamics.

Siam Quantum (SQ) is optimized for parallel computation and its capabilities include the calculation of Hartree-Fock and MP2 energies, minimum energy crossing point calculations, geometry optimization, population analysis, and quantum molecular dynamics.


\paragraph{Helper Applications}
FragIt \cite{Steinmann_2012} generates fragments of large molecules to use as input files in quantum chemistry programs that support fragment based methods.

cclib \cite{O_boyle_2008} provides a consistent interface for parsing and interpreting the results of a number of quantum chemistry packages. 

GaussSum \cite{O_boyle_2008} uses cclib to extract useful information from the results of quantum chemistry programs (ADF, GAMESS, Gaussian, Jaguar) including monitoring the progress of geometry optimization, the UV/IR/Raman spectra, molecular orbital (MO) levels and MO contributions.

Geac (Gaussian ESI Automated Creator) extracts data from Gaussian log files. 

Nancy\_EX \cite{Etienne_2014} post-processes Gaussian output and analyzes excited states including natural transition orbitals, detachment and attachment density matrices, and charge-transfer descriptors.

orbkit \cite{hermann2016orbkit} is a post-processing tool for the results of quantum chemistry programs.  It has native support for a number of programs (MOLPRO, TURBOMOLE, GAMESS-US, PROAIMS/AIMPAC, Gaussian) and additionally interfaces with cclib for additional file format support.  It can extract grid-based quantities such as molecular orbitals and electron density, as well as Muliken population charges and other properties.


\paragraph{Visualization}
CCP1GUI provides a graphical user interface to various computational chemistry codes with an emphasis on integration with the GAMESS-UK quantum chemistry program.

ccwatcher provides a graphical interface for the monitoring of computational chemistry programs.

Gabedit \cite{Allouche_2010} is a graphical user interface to a large number of quantum chemistry packages. It can create input files and graphically visualize calculation results.

J-ICE \cite{Canepa_2010} is a Jmol-based viewer for crystallographic and electronic properties that can be deployed as a Java applet embedded in a web browser.
 
QMForge provides a graphical user interface for analyzing and visualizing results of quantum chemistry DFT calculations (Gaussian, ADF, GAMESS, Jaguar, NWChem, ORCA, QChem).  Analyses include a number of population analyses, Mayer's bond order, charge decomposition, and fragment analysis.

\section{Cheminformatics}
 
Cheminformatics involves the representation, manipulation and analysis of molecular data \cite{Leach_2007,Gasteiger_2003}. Cheminformatic toolkits, although they may contain standalone utility programs, are primarily designed to function as libraries for other programs so that common functions, such as parsing molecular data, need not be reimplemented.  As libraries, the native programming language of a toolkit is particularly relevant as it influences the language programs that integrate with the toolkit can be written in.  In some cases, alternative language bindings, which essentially translate between programming languages, may be available, but due to the use of different idioms by different languages (e.g. object-oriented vs. functional, manual vs. automatic memory management) use of these non-native bindings may be cumbersome.  In addition to toolkits, we catalog standalone programs, including conformer generators for converted 2D information into 3D molecular structures (some of which have been critically evaluated \cite{Ebejer_2012}), graphical environments for creating and managing workflows, and molecular editors.